\subsection{Исследование точности анализа значений}
\subsubsection{Цель}
Точность анализа достижимости зависит от точности анализа значений.
Целью этого экспериментального исследования является оценка 
\subsubsection{Методика}
\begin{enumerate}
  \item Запуск анализатора, вычисление абстрактного состояния после
    каждой инструкции ветвления. Сбор трассы $\tilde{T}$.
  \item Инстументирование скрипта
    \begin{itemize}
      \item Преобразование исходного кода скрипта в абстрактное
	ситнаксическое дерево при помощи Closure Compiler.
      \item Добавление вызова специальной функции, записывающей в
	журнал значение всех переменных в текущей области видимости,
	после каждой инструкции цикла и ветвления
      \item Генерация исходного кода по измененному абстрактному
	синтаксическому дереву при помощи Closure Compiler.
    \end{itemize}
  \item Запуск инструментированного скрипта при помощи ringojs. Сбор
    трассы $T$.
  \item Сравнение трасс: вычисление метрик точного совпадения
    $\mu\left( T, \tilde{T} \right)$.
\end{enumerate}
\subsubsection{Исходные данные}
https://github.com/mootools/mootools-core

https://github.com/mootools/mootools-core-specs
\subsubsection{Результаты}

% \subsection{Исследование анализа достижимости}
% \subsubsection{Цель}
% \subsubsection{Методика}
% \begin{enumerate}
  % \item Запуск скрипта при помощи ringojs. Оценка времени выполнения.
  % \item Установка внутреннего счетчика времени ringojs в произвольное,
    % не превышающее времени выполнения, значение. При запуске ringojs с
    % установленным счетчиком, интерпретация скрипта приостановится
    % через заданное время.
  % \item Запуск скрипта при помощи ringojs с установленным счетчиком
    % времени. Пометка инструкции, на которой было остановлено
    % выполнение, как достижимой.
  % \item Инструментирование скрипта. Добавление специального комментария
    % перед инстуркцией, достижимость которой нужно проверить.
  % \item Запуск анализатора достижимости на инструментированном скрипте.
  % \item Оценка результатов анализа.
% \end{enumerate}
% Использовать ringojs \cite{ringojs}
% \subsubsection{Результаты}
