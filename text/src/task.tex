%
% Цели и задачи курсовой
%

Целью дипломной работы является построение статического анализаторa для программ
на языке \texttt{JavaScript}, который осуществлял бы контекстно-зависимый,
интерпроцедурный анализ \texttt{JavaScript}. 
\\[\smallskipamount]
К разрабатываемому анализатору предъявляются следующие требования:
\begin{itemize}
  \item Статический анализатор должен уметь отвечать на вопрос, достижима ли заданная
    точка программы, при каких значениях входных данных.
  \item Статический анализатор должен поддерживать все конструкции языка.
  \item Необходимо обосновать ограничения на используемые конструкции языка, при
    которых можно сделать обоснования полноты и точности анализа.
  \item Анализ должен быть консервативным. При этом для тех конструкций, которые
    существенно снижают точность анализа должны выводиться соответствующие
    предупреждения.
  \item Для статического анализатора должен быть построен набор автоматических
    функциональных тестов, проверяющих правильность его работы.
  \item Должно быть проведено экспериментальное исследование анализатора на
    задаче \emph{tainted mode}.
  \item В дипломной работе должен быть приведен краткий обзор близких работ.
\end{itemize}
% \\[\smallskipamount]
% Следует отметить, что задача о достижимости точки программы алгоритмически
% неразрешима. В данной работе описывается приближенное решение, поэтому одним из
% возможных ответов анализатора на вопрос о достижимости является
% \emph{``неизвестно''}.

% \begin{enumerate}
  % \item Разработать алгоритм, входными данными которого являются программа на
    % языке JavaScript и точка программы, а выходными данными являются 
    % \begin{itemize}
      % \item ответ на вопрос, может ли в заданной программе быть достигнута заданная
	% точка,
      % \item возможный путь в графе вызовов,
      % \item набор объектов, учавствовавших в определении потока управления,
      % \item набор ограничений, которым должны удовлетворять объекты.
    % \end{itemize}



