% Введение Содержит неформальную постановку, которая должна быть
% понятна неподготовленному читателю.  Именна эта часть описывает,
% какую задачу я решаю.  НЕ содержит результатов

Язык \texttt{JavaScript} создавался для написания программ,
добавляемых в текст веб-страницы и исполяемых внутри веб-браузера.
Внедрение программного интерфейса \texttt{DOM}, предоставляющего
программам удобный доступ к содержимому \texttt{HTML} документов, и
технологии \texttt{AJAX}, позволяющей асинхронно обмениваться данными
с веб-сервером, сделало язык самым используемым инструментом для
создания веб-приложений. Этот факт и последующая стандартизация
позволила языку проникнуть в области прикладного и серверного
программирования.

В результате широкого распространения выявились и подверглись критике
недостатки языка \cite{Crockford2008}. Двумя главными недостатками
являются низкая скорость выполнения, свойственная всем
интерпретируемым языкам, и проблемы безопасности, возникающие из-за
следующих свойств языка:
\begin{itemize} 
  \item В языке реализована прототип-ориентированная парадигма
    программирования, согласно которой, в отсутствует понятие класса,
    но существует понятие объекта. Объекты могут создаваться либо
    заново, перечислением своих методов и атрибутов, либо путём
    клонирования существующих. Для большей гибкости в язык добавлена
    возможность добавлять, модифицировать и удалять методы и поля
    объекта во время исполнения.
  \item
    Язык является интерпретируемым, c возможностью во время выполнения
    формировать произвольную текстовую строкy и, при помощи
    конструкции \texttt{eval}, интерпретировать её как программу,
    либо, при помощи конструкции \texttt{Function}, использовать её
    для определения тела функции.  
  \item В языке используется слабая
    динамическая типизация, допускаются неявные преобразование
    объектов любого типа.  
  \item В языке области видимости
    представлены в виде объектов. При запуске программы создается
    глобальный объект, методами которого являются библиотечные функции
    и функции взаимодействия с интерпретатором. Возможность
    переопределять методы объектов во время выполнения позволяет
    заместить вызов безопасной библиотечной функции произвольным
    кодом.
\end{itemize}

На текущий момент существует несколько подходов к решению проблемы
безопасности программ на \texttt{JavaScript}. Самым простым является
введение безопасного подмножества языка. Однако, из-за существенного
снижения возможностей языка, применимость этого подхода ограничена.
Другим подходом является инструментирование кода.  Добавление внешней
программы, следящей за выполнением набора условий в процессе
интерпретации \texttt{JavaScript} не всегда возможно и усугубляет
второю большую проблему языка -- скорость интерпретации. Еще одним
подходом является автоматической преобразование кода программы
написанной на более безопасном языке, например Java, в
\texttt{JavaScript}. Этот подход нельзя применить к уже существующим
програмам. Кроме того, он приводит к появлению избыточного и мёртвого
кода, что как и в предыдущем подходе, снижает скорость. 

В последнее десятилетие из-за роста популярности ``скриптовых''
языков, в том числе \texttt{JavaScript}, были предприняты попытки
использования средств статического анализа, хорошо зарекомендовавших
себя при решении проблем компилируемых и статически типизированных
языков, для решения проблемы безопасности динамических языков. 
При анализе \emph{JavaScript} авторы столкнулись с многочисленными
проблемами, возникающими из-за свойств языка, перечисленных в
следующей главе. Зачастую, чтобы произвести корректный анализ,
рассматривалось подмножество языка.
