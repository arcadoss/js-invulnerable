Критерии различия работ: полнота анализируемого языка (весь язык или
его подмножество), используемые методы статического анализа,
использование дополнительных средств для анализа, назначение.  \\

В работе \cite{jang2009points} реализован первый \emph{points-to}
анализ языка для его оптимизации. Авторы напрямую применили идеи,
использованные для анализа языка \texttt{С}. Анализ обладает следующими
свойствами:
\begin{itemize}
  \item 
    анализ интрапроцедурный, без учета потока управления и контекста
  \item 
    анализ производится на основе множеств ограничений
    \cite{heintze1992set}
\end{itemize}
Анализ производится для подмножества языка\cite{thiemann2005towards}. Из языка были полностью
убраны инструменты поддерживающие рефлексию. 
Для улучшения точности каждое свойство объекта рассматривается
отдельно. Это отличает его от анализа Андерсона, в котором массивы
рассматриваются как единые сущности. Пусть свойствам некоторого объекта
присваиваются некоторые значения, если обращаться к свойству как к
элементу словаря, используя в качестве ключа динамически созданную
строку, то можно изменить любое его свойство. Авторы перешли
ограничили анализ этой проблемой. \\

В работе \cite{jensen2009type} реализован \emph{корректный} анализ
типов переменных программы.  Анализатор типов применяется в средах
разработки для обнаружения ошибок на ранней стадии написания программ,
точного автодополнения и безопасного рефакторинга кода. Основным
методом анализа является абстрактная интерпретация и вычисление
неподвижной точки в монотонной структуре\cite{Nielson1999}. Высокая
точность анализа достигается за счет создания подробной модели при
абстрактной интрепретации и набора дополнительных техник:
\begin{itemize}
  \item 
    \emph{recency abstraction} \cite{Balakrishnan2006}
  \item 
    \emph{lazy propagation} \cite{jenseninterprocedural}
  \item 
    интерпроцедурный анализ с учетом контекста и потока управления
  \item 
    абстрактная сборка мусора \cite{Might2006} 
  \item 
    выполнение \emph{reference-to} анализа в процессе определения
    типов
\end{itemize}
В этой работе рассматривается весь язык, а не его подмножество. 
\\

В работе \cite{guarnieri2009gatekeeper} предложен способ усиления
безопасности программ на JavaScript. Для этого вводится набор политик,
определяющих нежелательное поведение программы, например запрещающих
переопределение встроенных функций. Нежелательное поведение выражается
на специальном языке, оперирующим фактами, полученными в ходе
статического анализа и выполнения инструментирования. Авторами
реализован \emph{корректный} points-to анализ программ из подмножества
языка. На язык наложены следующие ограничения:
\begin{enumerate}
  \item 
    Язык не содержит инструкций обеспечивающих рефлексию -- запрещены
    инструкции \texttt{eval} и \texttt{Function}.
  \item 
    Запрещено изменение свойства \texttt{innerHtml} элементов
    содержащихся в документе. В случае, когда исполняемый скрипт
    встроем в веб-страницу, при помощи интерфейса \texttt{DOM} можно
    получить доступ к объектам, соответствующим элементам
    веб-страницы. Свойство элемента \texttt{innerHtml} определяет код
    которым будет замещен этот элемент при отображении страницы. 
  \item 
    Функция не является объектом первого класса (запрещается создавать
    тело функции во время выполнения)
  \item	
    Язык не содержит инструкции \emph{with}, позволяющей временно
    изменять цепочку областей видимости
\end{enumerate}
Инструментирование кода было использовано для выражения дополнительных
требований и как альтернатива ограничению 2, являющемуся очень
сильным.
